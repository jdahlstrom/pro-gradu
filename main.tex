%
% Document template suitable for use as a latex master-file for masters
% thesis in University of Turku Department of Information Technology. 
% Relies on itpackage.sty for additional definitions.
%
% Sami Nuuttila (samnuutt@utu.fi) 
%
% Last mod 31.10.2013:
% 
% Why:
%  � No need for anyone to invent the wheel again. You can of course do that
%    if you wish - TeX'll even let you invent many different kinds of wheels
%  � That said, if you come up with a great new wheel I'd like to hear about 
%    it - it might even end up being used here 
% 
% Features:
%  � Proper page sizes as required by university guide for students:
%      � proper font sizes as well as linespacings
%      � proper size of margins
%  � Generic title page:
%      � \gentitle
%  � Generic abstract page(s):
%      � \begin{itabstract}{Keywords}
%          abstract text
%        \end{itabstract}
%      � \begin{ittiivis}...\end{ittiivis} provides finnish version
%         � ittiivis defaults to finnish so no need to issue 
%           \selectlanguage{finnish}
%      � total number of pages as well as total number of pages in appendices
%        are automagically handled
%  � Entry environment:
%      � \begin{entry}[widest label]
%          \item[1st label text] ...
%          \item[2nd label text] ...
%        \end{entry}
%      � the actual items are aligned to suit the widest label, which is
%        given as an argument to the environment
%  � Use of specific latex packages to ease in formatting the thesis:
%      � format table of contents to have bibliography shown as references
%        as well as other fixes           (tocbibind)
%      � enhanced verbatim handling       (sverb)
%      � source code inclusion            (listings)
%      � handling of headers and footers  (fancyhdr)
%
%      � consultation of the manuals of these packages is strongly
%        encouraged 
%
% Assumptions:
%  � itpackage.sty file is available 
%  � each chapter is as a separate file which is read in with e.g. \input
%   
% Miscellaneous:
%  � comments are welcome
%  � should a required package be missing see http://www.ctan.org/ 
%  � http://www.ctan.org/tex-archive/info/lshort/english/lshort.pdf
%
%%%%%%%%%%%%%%%%%%%%%%%%%%%%%%%%%%%%%%%%%%%%%%%%%%%%%%%%%%%%%%%%%%%%%%%%%%%

%%%%%%%%%%%%%%%%%%%%%%%%%%%%%%%%%%%%%%%%%%%%%%%%%%%%%%%%%%%%%%%%%%%%%%%%%%%
%
% load all required packages
%
%%%%%%%%%%%%%%%%%%%%%%%%%%%%%%%%%%%%%%%%%%%%%%%%%%%%%%%%%%%%%%%%%%%%%%%%%%%

% document is based on report class
\documentclass[a4paper,12pt]{report}

% load ams-packages for maths
\usepackage{amssymb,amsthm,amsmath}

% load babel-package for automatic hyphenation
\usepackage[english,finnish]{babel}
\usepackage[utf8]{inputenc}

% load graphicx package
%   � automagically select proper parameters depending on whether
%     we're running pdflatex or latex
%   � specify \includegraphics{file} without the file extension
%     (.eps /.pdf (/ .png / .jpeg / ...)), tex should select the proper file
%
\usepackage{ifpdf}
\usepackage{graphicx}
%% !!! NOTE: if you have ancient LaTeX distribution then you might need to
%% use the following instead
%% \newif\ifpdf
%% \ifx\pdfoutput\undefined
%%   \pdffalse
%% \else
%%   \pdfoutput=1
%%   \pdftrue
%% \fi
%% % if graphicx complains about option clash remove the [pdftex] option
%% \ifpdf
%%   \usepackage[pdftex]{graphicx}
%% \else
%%   \usepackage[dvips]{graphicx}
%% \fi


% load tocbibind package 
%   � do not include table of contents in itself
%   � convert the name of bibliography to references
\usepackage[nottoc]{tocbibind}
\settocbibname{References}

% load sverb package
%   � enhanced handling of verbatim stuff; listing environment
\usepackage{sverb}

% load listings package
%   � handle inclusion of source code
\usepackage{listings}

% load fancyheaders package
%   � the actual headers and footers are set later
\usepackage{fancyhdr}

% load itpackage 
%   � additional defines and stuff
\usepackage{itpackage}

\usepackage{setspace}

\usepackage{url}

\usepackage{listings}
\lstset{
    basicstyle=\small\ttfamily\singlespacing,
    numbers=left,
    numberstyle=\tiny,
    frame=tb,
    columns=fullflexible,
    showstringspaces=false
}
\lstdefinelanguage{JavaScript}{
  keywords={break, case, catch, continue, debugger, default, delete, do, else, false, finally, for, function, if, in, instanceof, new, null, return, switch, this, throw, true, try, typeof, var, void, while, with},
  morecomment=[l]{//},
  morecomment=[s]{/*}{*/},
  morestring=[b]',
  morestring=[b]",
  sensitive=true
}

\usepackage{float}

\floatstyle{plain} % optionally change the style of the new float
\newfloat{code}{b}{myc}

% uncomment the following snippet to get rid of Luku/Chapter text at the
% beginning of each Chapter... 
%\makeatletter
%\renewcommand{\@chapapp}{\relax}
%\renewcommand{\@makechapterhead}[1]{%
%  \vspace*{50\p@}%
%  {\parindent \z@ \raggedright \normalfont
%    \ifnum \c@secnumdepth >\m@ne
%        \huge\bfseries \@chapapp\space \thechapter\space\space
%    \fi
%    \interlinepenalty\@M
%    \Huge \bfseries #1\par\nobreak
%    \vskip 40\p@
%  }}
%\makeatother

% custom commands
\newcommand{\ab}[1]{\textsc{#1}}

% ugly, needs another font
% \newcommand{\num}[1]{\oldstylenums{#1}}
\newcommand{\num}[1]{#1}

%%%%%%%%%%%%%%%%%%%%%%%%%%%%%%%%%%%%%%%%%%%%%%%%%%%%%%%%%%%%%%%%%%%%%%%%%%%
%
% main document starts here
%
%%%%%%%%%%%%%%%%%%%%%%%%%%%%%%%%%%%%%%%%%%%%%%%%%%%%%%%%%%%%%%%%%%%%%%%%%%%
\begin{document}

% select language here by removing comment from one of the two rows below
% \selectlanguage{finnish}\fintrue
\selectlanguage{english}\def\enclname{Appendices}\finfalse

%%% fill in your information below
\workinfo{Johannes Dahlström}
    {Reactive User Interfaces in the Web}
    {First Supervisor}
    {Second Supervisor}
    {2015}
    {Month}
    {Kuukausi} 
    
% set the type of your thesis (Diplomityö, TkK -tutkielma, etc.) and
% laboratory or field of study below
\worktype{M.Sc. Thesis}{Pro gradu} 
\deptinfo{Computer Science}{Tietojenkäsittelytiede}


% generate the title page 
\gentitle

% generate tiivistelma 
\begin{ittiivis}{tähän, lista, avainsanoista}
Tarkempia ohjeita tiivistelmäsivun laadintaan löytyy opiskelijan
yleisoppaasta, josta alla lyhyt katkelma.
\end{ittiivis}

% if your thesis is in english then this is also required (is it???)
\begin{itabstract}{list, of, keywords}
If your thesis is in english this might also be required???
\end{itabstract}


% empty pagestyle for table of contents etc. 
%
% the redefinition of plain style works also for 1st pages of chapters,
% which is the default in report class. Just state \thispagestyle{empty}
% after \chapter{something} if you want empty style for the 1st pages. 
%
\pagestyle{empty}
\fancypagestyle{plain}{
  \fancyhf{}
  \renewcommand{\headrulewidth}{0 pt}
}

% roman numbering for table of contents etc.
\pagenumbering{roman}

% insert table of contents, list of figures, list of tables
%
% setting the counter to zero effectively removes the page number from
% the toc, lof etc. entries in the toc since there is no roman numeral
% for zero ;-) (if you want them without numbering)
%
%\setcounter{page}{0}
\tableofcontents
\clearpage
%\setcounter{page}{0}
\listoffigures 
\clearpage
%\setcounter{page}{0}
\listoftables

% possibly insert 'list of acronyms'
%
%   � create a chapter called List Of Acronyms (or whatever), which
%     should contain all your acronym definitions, e.g. 
%     \chapter{List Of Acronyms} 
%   � the secnumdepth trickery is needed because acronyms are as a
%     standard chapter and we are faking '\listofacronyms'
%
%\setcounter{secnumdepth}{-1}
%\input{your acronym chapter's file name}
%\setcounter{secnumdepth}{2}


% setup page numbering, page counter, etc.
\startpages

%%%%%%%%%%%%%%%%%%%%%%%%%%%%%%%%%%%%%%%%%%%%%%%%%%%%%%%%%%%%%%%%%%%%%%%%%%%
%
% hi ho hi ho it's off to work we go....
%
% from now on you're on your own -- good luck!
%
% that is to say that the actual thesis should come here 
%
%%%%%%%%%%%%%%%%%%%%%%%%%%%%%%%%%%%%%%%%%%%%%%%%%%%%%%%%%%%%%%%%%%%%%%%%%%%

\chapter{Introduction}

Interactive software systems are becoming progressively more complex, driven by the demand for richer user interaction along with the ever-increasing multimedia capabilities of modern hardware. Applications are increasingly targeted on the World Wide Web platform, and the inherently distributed nature of web applications brings forth its own challenges. Not only are they required to serve an ever-increasing number of users, but also, increasingly, to facilitate interaction \emph{between} users. 

In object-oriented programming, the usual means of implementing a user interface is to make use of the Observer design pattern. In this pattern, interested objects can register themselves as \emph{observers}, or listeners, of events, such as mouse clicks or key presses, sent by user interface elements. When an observer is notified of an event, it reacts in an appropriate manner by initiating a computation or otherwise changing the application state. Similarly, the user interface may react to events triggered by some underlying computation, signaling the user that something requires his or her attention.

The traditional observer pattern has several disadvantages. Observers are difficult to compose and reuse, leading to instances of partial or complete code duplication. Events often lack important context, making it necessary to manually keep track of the program state and share the bookkeeping between different observers. The resulting code typically forms a complex and fragile state machine, making it difficult to reason about its behavior and correctness.

Reactive programming is a programming style centered on the concept of propagating change. In a reactive system, a variable can be bound to other variables so that its value changes automatically as a response to value changes in other components of the bound system. A common example of such reactivity is a spreadsheet application, where the value of a cell can be a formula referring to several other cells. The displayed value of the cell is refreshed whenever the value of a referenced cell changes.

Reactive functional programming

In the recent years, several mainstream object-oriented programming languages used in the industry have adopted concepts traditionally belonging to the relatively academic realm of functional programming. These include functions as first-class values; anonymous functions (lambdas); and combinators such as map, filter, and reduce. One impetus for this paradigm shift has been the constantly increasing importance of concurrency and parallelism in software. Correctly managing and reasoning about mutable state shared between concurrent threads of execution is notoriously difficult, and the general disposition towards immutable state in functional programming has proven to be a useful basis for building better concurrency abstractions.

One way of improving upon the observer pattern is to elevate the abstract concept of an event stream to a first-class data type. Event streams, or \emph{observables}, can then be merged, transformed, and otherwise manipulated using the familiar functional toolset.

Vaadin is a web application framework written in Java, aiming to provide a rich set of user interface components facilitating rapid application development. It also contains a data binding layer for propagating input and output between the user and a data model. Both the user interface and data binding are designed in terms of the observer pattern. 
\chapter{Programming Web Applications}

\section{A Brief Introduction to the Web}

The World Wide Web, or the Web for short, is a massive distributed information system in the Internet. It constitutes a large set of interlinked \emph{hypertext documents}, accessed using special browser software.

The Web project was initiated by the British computer scientist Timothy Berners-Lee while working at the European Organization for Nuclear Research, or \textsc{cern}, in the late 1980s and early 1990s. His original aim was to improve the information sharing between scientists, but the Web soon expanded to more general use.

Hypertext, one of the central concepts of the Web, refers to electronic documents linked to other documents via embedded references called hyperlinks. An early vision of the concept was presented in the seminal 1945 essay \emph{As We May Think} by Vannevar Bush. In the 1960s, the term ``hypertext'' was coined by Ted Nelson, and a working hypertext system was showcased by Douglas Engelbart in the famous "Mother of All Demos".

Hypertext documents in the Web are written in \textsc{html} (Hypertext Markup Language) that is based on the older \textsc{sgml} (Standardized General Markup Language). A \textsc{html} document is a structured text file consisting of a hierarchy of nested elements, representing the logical and visual structure of the document. A Web browser interprets the \textsc{html} and renders a graphical representation of it to the user, managing layouting, typesetting, multimedia, and possible interactivity as specified in the document.

The Web is based on a client-server architecture utilizing \textsc{http} (Hypertext Transfer Protocol). A client application, typically a Web browser, connects to a \textsc{http} server, requesting a \emph{resource} such as a Web page, an image, or other type of file. Resources are identified via unique textual identifiers, \textsc{URI}s.

A single server can attend to several clients---constrained by the available resources---but the clients cannot communicate directly with each other. The server must work as a mediator in any client-to-client interaction. Another limitation in the \textsc{http} model is that the server cannot proactively send messages to the clients; it may only respond to requests initiated by them. Furthermore, the protocol is stateless; two consecutive requests by the same user are independent and not associated with the same session without separate bookkeeping.

A Web resource need not be a physical file served from mass storage; the server may instead elect to generate the response partially or fully programmatically. Thus, when reloading a page, its content may change dynamically without manual maintenance. For instance, the server might do a database query and present the results to the client as a \textsc{html} document. In practice, it is useful to modularize a Web server so that it can delegate request handling to a set of subprograms on a request-by-request basis. 

A simple protocol, \emph{Common Gateway Interface} (\textsc{CGI}), was developed to facilitate such delegation from a Web server to an auxiliary program. Early CGI applications were typically written in C or Perl. 

In the early days of the Web, the only form of interaction between the user and the Web server was requesting pages either by typing an explicit \textsc{uri} or following hyperlinks. Use cases soon emerged for the ability for the user to send input data to the server; the latter in turn serving another page based on the user input and possibly save the input to persistent storage for future use. In 1993, Mosaic, one of the first graphical browsers, added to its \textsc{html} dialect a rudimentary set of input elements, including text fields, buttons, and list boxes. These elements were later included in the \textsc{html} 2 standard.

Compared to regular desktop applications, this rudimentary interactivity was rather slow and awkward. To process any user input, the browser would have to send a \textsc{http} request to the server, where it would be processed and a new Web page, generated based on the input, served to the client. Fetching up-to-date content from the server would require the user to manually ask the browser to refresh the page.

For a more fulfilling interaction, a client-side programming model was required. \emph{JavaScript}, an interpreted language executed by the browser, was developed in 1995 by Brendan Eich. It was originally included in the Netscape Navigator browser. The language could be used to dynamically manipulate the document, typically interactively as a response to input events such as mouse clicks and key presses. For security reasons, JavaScript code in a browser is executed in a "sandbox", a secure virtual machine, and the language initially offered practically no access to standard operating system services such as the file system or the network. Partially due to these limitations, client-side scripting was considered by many a novelty at best and a nuisance at worst.

To better utilize the distributed aspect of the Web in client-side programming, a method for making programmatic \textsc{html} requests, without necessitating the reloading the whole page---and losing all client-side state---was required. Such functionality was developed by Microsoft in the late 1990s and became known as Ajax (Asynchronous JavaScript and \textsc{xml}). It is arguable that Ajax was the technology that started the ongoing age of Web applications. Despite the name, \textsc{xml} is not necessarily used as the message format. Again as a security measure, Ajax connections are by default limited to the server from which the page itself was requested.

As the name implies, Ajax requests are \emph{asynchronous}. Because JavaScript programs are single-threaded, they cannot simply wait for the response without pausing user interaction. Instead, the browser initiates the request in the background and notifies the script of its eventual completion.

\emph{\textsc{html} 5} is a common name for various technologies that aim to improve the capabilities and richness of interaction of Web applications. Several of these new features are programming interfaces that expose operating system services to JavaScript in a controlled fashion.

The Web has been transformed from a simple document retrieval system to a full-fledged distributed application platform. Despite their potentially awkward user interfaces, Web applications have several advantages. They do not require a separate installation step, they run on any platform with a modern Web browser, and they are intrinsically network-aware, permitting interaction with not just the server, but also other users connected to the same server.



\begin{lstlisting}[language=HTML]
<!DOCTYPE html5>
<html>
  <head>
    <title>Test</title>
  </head>
  <body>
    test
  </body>
</html>
\end{lstlisting}

\begin{lstlisting}[language=JavaScript]
var btn = document.getElementById("button");

btn.addEventListener("click", function () {
    alert("Clicked a button!");
});
\end{lstlisting}

\section{Web Application Architecture}

\section{User Interface Programming}

\subsection{Widgets}

\subsection{Events and Observers}

\section{Servlets}

\section{Vaadin}

\subsection{Client-Server Architecture}

\subsection{Components}

\subsection{Events}

\subsection{Data Model}

\chapter{Reactive Programming}

\section{Reactive Primitives}

\subsection{Behaviors and Events}

\subsection{Futures and Promises}

\subsection{Observables}

\subsection{Async and Await}

\subsection{Actors}

\subsection{Composability}

\section{Reactive libraries}

\subsection{Rx}

\subsection{React.js}
\chapter{Case Study: Reactive Vaadin}
\chapter{Validation}
\chapter{Conclusions}

% insert references
%  � included unsrtf.bst provides finnish language version of unsrt
%    style, change below if needed
\bibliographystyle{unsrt}
\bibliography{2.bib}
\nocite{*}

%%%%%%%%%%%%%%%%%%%%%%%%%%%%%%%%%%%%%%%%%%%%%%%%%%%%%%%%%%%%%%%%%%%%%%%%%%%
%
% Almost there....
%
%%%%%%%%%%%%%%%%%%%%%%%%%%%%%%%%%%%%%%%%%%%%%%%%%%%%%%%%%%%%%%%%%%%%%%%%%%%

% make sure pagecount is correct even if references overflow to a new page
\pagebreak\addtocounter{page}{-1}
\eofpages
\appendices

% create your appendix chapters with command \appchapter{some name} instead
% of \chapter{some name} for the automagic page counting to work
%\input{file name of appchapter xxx}
%\input{file name of appchapter yyy}
%\input{file name of appchapter zzz}
%\input{and so on}

%%%%%%%%%%%%%%%%%%%%%%%%%%%%%%%%%%%%%%%%%%%%%%%%%%%%%%%%%%%%%%%%%%%%%%%%%%%
%
% main document ends here
%
%%%%%%%%%%%%%%%%%%%%%%%%%%%%%%%%%%%%%%%%%%%%%%%%%%%%%%%%%%%%%%%%%%%%%%%%%%%
\eofapppages
\end{document}


