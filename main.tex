%
% Document template suitable for use as a latex master-file for masters
% thesis in University of Turku Department of Information Technology. 
% Relies on itpackage.sty for additional definitions.
%
% Sami Nuuttila (samnuutt@utu.fi) 
%
% Last mod 31.10.2013:
% 
% Why:
%  � No need for anyone to invent the wheel again. You can of course do that
%    if you wish - TeX'll even let you invent many different kinds of wheels
%  � That said, if you come up with a great new wheel I'd like to hear about 
%    it - it might even end up being used here 
% 
% Features:
%  � Proper page sizes as required by university guide for students:
%      � proper font sizes as well as linespacings
%      � proper size of margins
%  � Generic title page:
%      � \gentitle
%  � Generic abstract page(s):
%      � \begin{itabstract}{Keywords}
%          abstract text
%        \end{itabstract}
%      � \begin{ittiivis}...\end{ittiivis} provides finnish version
%         � ittiivis defaults to finnish so no need to issue 
%           \selectlanguage{finnish}
%      � total number of pages as well as total number of pages in appendices
%        are automagically handled
%  � Entry environment:
%      � \begin{entry}[widest label]
%          \item[1st label text] ...
%          \item[2nd label text] ...
%        \end{entry}
%      � the actual items are aligned to suit the widest label, which is
%        given as an argument to the environment
%  � Use of specific latex packages to ease in formatting the thesis:
%      � format table of contents to have bibliography shown as references
%        as well as other fixes           (tocbibind)
%      � enhanced verbatim handling       (sverb)
%      � source code inclusion            (listings)
%      � handling of headers and footers  (fancyhdr)
%
%      � consultation of the manuals of these packages is strongly
%        encouraged 
%
% Assumptions:
%  � itpackage.sty file is available 
%  � each chapter is as a separate file which is read in with e.g. \input
%   
% Miscellaneous:
%  � comments are welcome
%  � should a required package be missing see http://www.ctan.org/ 
%  � http://www.ctan.org/tex-archive/info/lshort/english/lshort.pdf
%
%%%%%%%%%%%%%%%%%%%%%%%%%%%%%%%%%%%%%%%%%%%%%%%%%%%%%%%%%%%%%%%%%%%%%%%%%%%

%%%%%%%%%%%%%%%%%%%%%%%%%%%%%%%%%%%%%%%%%%%%%%%%%%%%%%%%%%%%%%%%%%%%%%%%%%%
%
% load all required packages
%
%%%%%%%%%%%%%%%%%%%%%%%%%%%%%%%%%%%%%%%%%%%%%%%%%%%%%%%%%%%%%%%%%%%%%%%%%%%

% document is based on report class
\documentclass[a4paper,12pt]{report}

% load ams-packages for maths
\usepackage{amssymb,amsthm,amsmath}

% load babel-package for automatic hyphenation
\usepackage[english,finnish]{babel}
\usepackage[utf8]{inputenc}

% load graphicx package
%   � automagically select proper parameters depending on whether
%     we're running pdflatex or latex
%   � specify \includegraphics{file} without the file extension
%     (.eps /.pdf (/ .png / .jpeg / ...)), tex should select the proper file
%
\usepackage{ifpdf}
\usepackage{graphicx}
%% !!! NOTE: if you have ancient LaTeX distribution then you might need to
%% use the following instead
%% \newif\ifpdf
%% \ifx\pdfoutput\undefined
%%   \pdffalse
%% \else
%%   \pdfoutput=1
%%   \pdftrue
%% \fi
%% % if graphicx complains about option clash remove the [pdftex] option
%% \ifpdf
%%   \usepackage[pdftex]{graphicx}
%% \else
%%   \usepackage[dvips]{graphicx}
%% \fi


% load tocbibind package 
%   � do not include table of contents in itself
%   � convert the name of bibliography to references
\usepackage[nottoc]{tocbibind}
\settocbibname{References}

% load sverb package
%   � enhanced handling of verbatim stuff; listing environment
\usepackage{sverb}

% load listings package
%   � handle inclusion of source code
\usepackage{listings}

% load fancyheaders package
%   � the actual headers and footers are set later
\usepackage{fancyhdr}

% load itpackage 
%   � additional defines and stuff
\usepackage{itpackage}

% uncomment the following snippet to get rid of Luku/Chapter text at the
% beginning of each Chapter... 
%\makeatletter
%\renewcommand{\@chapapp}{\relax}
%\renewcommand{\@makechapterhead}[1]{%
%  \vspace*{50\p@}%
%  {\parindent \z@ \raggedright \normalfont
%    \ifnum \c@secnumdepth >\m@ne
%        \huge\bfseries \@chapapp\space \thechapter\space\space
%    \fi
%    \interlinepenalty\@M
%    \Huge \bfseries #1\par\nobreak
%    \vskip 40\p@
%  }}
%\makeatother

%%%%%%%%%%%%%%%%%%%%%%%%%%%%%%%%%%%%%%%%%%%%%%%%%%%%%%%%%%%%%%%%%%%%%%%%%%%
%
% main document starts here
%
%%%%%%%%%%%%%%%%%%%%%%%%%%%%%%%%%%%%%%%%%%%%%%%%%%%%%%%%%%%%%%%%%%%%%%%%%%%
\begin{document}

% select language here by removing comment from one of the two rows below
% \selectlanguage{finnish}\fintrue
\selectlanguage{english}\def\enclname{Appendices}\finfalse

%%% fill in your information below
\workinfo{Johannes Dahlström}
{Reactive User Interfaces in Web Applications}
{First Supervisor}
{Second Supervisor}
{2015}
{Month}
{Kuukausi} 
% set the type of your thesis (Diplomityö, TkK -tutkielma, etc.) and
% laboratory or field of study below
\worktype{Pro gradu}{Pro gradu} 
\deptinfo{Computer science}{Tietojenkäsittelytiede}


% generate the title page 
\gentitle

% generate tiivistelma 
\begin{ittiivis}{tähän, lista, avainsanoista}
Tarkempia ohjeita tiivistelmäsivun laadintaan löytyy opiskelijan
yleisoppaasta, josta alla lyhyt katkelma.
\end{ittiivis}

% if your thesis is in english then this is also required (is it???)
\begin{itabstract}{list, of, keywords}
If your thesis is in english this might also be required???
\end{itabstract}


% empty pagestyle for table of contents etc. 
%
% the redefinition of plain style works also for 1st pages of chapters,
% which is the default in report class. Just state \thispagestyle{empty}
% after \chapter{something} if you want empty style for the 1st pages. 
%
\pagestyle{empty}
\fancypagestyle{plain}{
  \fancyhf{}
  \renewcommand{\headrulewidth}{0 pt}
}

% roman numbering for table of contents etc.
\pagenumbering{roman}

% insert table of contents, list of figures, list of tables
%
% setting the counter to zero effectively removes the page number from
% the toc, lof etc. entries in the toc since there is no roman numeral
% for zero ;-) (if you want them without numbering)
%
%\setcounter{page}{0}
\tableofcontents
\clearpage
%\setcounter{page}{0}
\listoffigures 
\clearpage
%\setcounter{page}{0}
\listoftables

% possibly insert 'list of acronyms'
%
%   � create a chapter called List Of Acronyms (or whatever), which
%     should contain all your acronym definitions, e.g. 
%     \chapter{List Of Acronyms} 
%   � the secnumdepth trickery is needed because acronyms are as a
%     standard chapter and we are faking '\listofacronyms'
%
%\setcounter{secnumdepth}{-1}
%\input{your acronym chapter's file name}
%\setcounter{secnumdepth}{2}


% setup page numbering, page counter, etc.
\startpages

%%%%%%%%%%%%%%%%%%%%%%%%%%%%%%%%%%%%%%%%%%%%%%%%%%%%%%%%%%%%%%%%%%%%%%%%%%%
%
% hi ho hi ho it's off to work we go....
%
% from now on you're on your own -- good luck!
%
% that is to say that the actual thesis should come here 
%
%%%%%%%%%%%%%%%%%%%%%%%%%%%%%%%%%%%%%%%%%%%%%%%%%%%%%%%%%%%%%%%%%%%%%%%%%%%

\chapter{Introduction}


%\input{file name of chapter yyy}
%\input{file name of chapter zzz}
%\input{and so on...}

% insert references
%  � included unsrtf.bst provides finnish language version of unsrt
%    style, change below if needed
\bibliographystyle{unsrt}
%\bibliography{your bibliography's file name}

%%%%%%%%%%%%%%%%%%%%%%%%%%%%%%%%%%%%%%%%%%%%%%%%%%%%%%%%%%%%%%%%%%%%%%%%%%%
%
% Almost there....
%
%%%%%%%%%%%%%%%%%%%%%%%%%%%%%%%%%%%%%%%%%%%%%%%%%%%%%%%%%%%%%%%%%%%%%%%%%%%

% make sure pagecount is correct even if references overflow to a new page
\pagebreak\addtocounter{page}{-1}
\eofpages
\appendices

% create your appendix chapters with command \appchapter{some name} instead
% of \chapter{some name} for the automagic page counting to work
%\input{file name of appchapter xxx}
%\input{file name of appchapter yyy}
%\input{file name of appchapter zzz}
%\input{and so on}

%%%%%%%%%%%%%%%%%%%%%%%%%%%%%%%%%%%%%%%%%%%%%%%%%%%%%%%%%%%%%%%%%%%%%%%%%%%
%
% main document ends here
%
%%%%%%%%%%%%%%%%%%%%%%%%%%%%%%%%%%%%%%%%%%%%%%%%%%%%%%%%%%%%%%%%%%%%%%%%%%%
\eofapppages
\end{document}


